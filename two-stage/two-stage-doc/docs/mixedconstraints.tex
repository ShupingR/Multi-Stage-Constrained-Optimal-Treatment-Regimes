\documentclass[14pt]{extreport}       
\usepackage[a4paper, total={7.6in, 9.8in}]{geometry}            
\usepackage[utf8]{inputenc}                            
\usepackage[T1]{fontenc}                              
\usepackage{amsmath}                            
\usepackage{amsfonts}                            
\usepackage{amssymb}                 
\usepackage{enumitem}           
\usepackage{amssymb}
\usepackage{xfrac}
\let\oldemptyset\emptyset
\let\emptyset\varnothing
\begin{document}
\setlength\parindent{0pt}
\textbf{The original constrained problem}\\

The original constrained problem is stated as 
\begin{flalign*}
\min_{\tau} & \iint -\text{sgn}\left( v \right) u \,f_{Y}\left( u,v ;\tau \right) \,du\,dv \\
\text{subject to } & \kappa - \iint \text{sgn}\left( v \right) w\,f_{Z}\left( w,v;\tau \right)\, dw\,dv \geq 0 , \text{and } \tau^{\intercal} \tau = 1.
\end{flalign*}
Suppose that the strict feasible set strict($\mathcal{F}$) is non-empty, and let $\tau^0$ denote a constrained minimizer of this original problem. For simplicity, we let $g(\tau) = -\iint \text{sgn}(v)u\,f_Y(u, v; \tau)\,du \,dv$, $c_1(\tau) = \kappa - \iint \text{sgn}(v) w f_Z(w, v;\tau)\,dw\,dv$, and $c_2(\tau) = \tau^{\intercal}\tau$. The value of $g(\tau)$ at $\tau = \tau^0$, $g(\tau^0)$, is denoted by $g^0$. Similarly, $c_i^0$ denotes the value of $c_i(\tau)$ at $\tau = \tau^0$, $c_i(\tau^0)$, for $i = 1, 2$. Also, $\mathcal{A}^0$ denotes the set of active constraint at $\tau^0$,  $\mathcal{A}(\tau^0)$. In our current case, it is either $\mathcal{A}^0 =\{ c_1^0, \, c_2^0 \}$, or $\mathcal{A}^0 = \{ c^0_2\}$.\\

\textbf{Penalty-barrier function}\\

One of the method to solve equality-inequality mixed constrained optimization problem is use penalty-barrier function, which is a composite measure of the objective function and the penalty of violating the constraints.  The penalty-barrier function is then formalized as 
\begin{gather*}
\begin{flalign*}
\Phi_{BP}(\tau, \mu) 
= & \iint - \text{sgn}(v) u f_Y(u, v; \tau)\,du\,dv\\
& - \mu \text{ ln} \left[ \kappa - \iint \text{sgn}(v) w f_{Z} (w, v; \tau) \,dw \,dv \right] + \frac{1}{2\mu} \| \tau^{\intercal}\tau -1 \|_2^2 ,
\end{flalign*}
\end{gather*}
where $\mu$ is a sequence of positive decreasing small constants converging to zero.  An unconstrained minimizer of  $\Phi_{BP}(\tau, \mu)$ is denoted by $\tau^*(\mu)$ for emphasizing that it is a vector function of $\mu$, or $\tau^*_{\mu}$ for short. It can be proven that the constraint is strictly satisfied, i.e., $c(\tau^*_{\mu}) = \kappa - \iint \text{sgn}\left( v \right) w\,f_{Z}\left( w,v;\tau^*_{\mu} \right)\, dw\,dv > 0$.\\

There is theorem which gives the conditions under which, for sufficiently small $\mu$, the sequence $\{\tau_{\mu}^*\}$ defines a differentiable penalty-barrier trajectory converging to $\tau_{\mu}^0$.\\

To find $\tau_{\mu}^*$, we exploit its stationarity. The gradient of $\Phi_{PB}(\tau, \mu)$ is 
\begin{gather*}
\begin{flalign*}
\nabla_{\tau} \Phi_{PB}(\tau, \mu) = &\iint - \text{sgn}(v) u \nabla_{\tau} f_Y(u, v; \tau)\,du\,dv \\
&+ \mu \, \frac{\iint \text{sgn}(v) w \nabla_{\tau} f_{Z} (w, v; \tau) \,dw \,dv}{\kappa - \iint \text{sgn}(v) w f_{Z} (w, v; \tau) \,dw \,dv} + \frac{2}{\mu}\,(\tau^{\intercal}\tau-1)\tau,
\end{flalign*}
\end{gather*}
noting that $\nabla_{\tau}$ represents the first order derivative with respect to $\tau$.
If we are willing to assume that $\Phi_{PB}(\tau, \mu)$ is twice-continuously differentiable, it must hold that $\nabla \Phi_{PB}(\tau^*_{\mu}, \mu) = 0$ to satisfy the stationarity, i.e.,
\begin{flalign*}
&\iint \text{sgn}\left( v \right) u \, \nabla_{\tau} f_{Y}\left( u,v ;\tau^*_{\mu} \right) \,du\,dv  
=   \mu \, \frac{\iint \text{sgn}(v) w \nabla_{\tau} f_{Z} (w, v; \tau^*_{\mu}) \,dw \,dv}{\kappa - \iint \text{sgn}(v)\,w\,f_{Z} (w, v; \tau^*_{\mu}) \,dw \,dv} + \frac{2}{\mu}\, (\tau^{*\intercal}_{\mu}\tau^*_{\mu}-1)\tau^*_{\mu}, 
\end{flalign*}
with $\tau^{*\intercal}_{\mu}\tau^{*\intercal}_{\mu} - 1=0$. The barrier multiplier, the coefficient in this linear relationship above, denoted by $\lambda_{\mu}$, is defined as 
\begin{flalign*}
\lambda_{\mu} \triangleq \frac{\mu}{\kappa - \iint \text{sgn}(v) w f_{Z}(w, v; \tau_{\mu}) \,dw \,dv}.
\end{flalign*}
This relationship can be re-written as
\begin{flalign*}
\lambda_{\mu}\left[ \kappa - \iint \text{sgn}(v) w f_{Z}(w, v; \tau_{\mu}) \,dw \,dv \right]= \mu.
   \end{flalign*}
This relationship between the barrier multiplier, the constraint value, and the barrier parameter, called perturbed complementarity, is analogous as $\mu \to 0$ to the complementarity condition $c(\tau^*) \lambda^* = 0$ that holds at a KKT point.\\



\textbf{Estimation of the log-barrier penalty function}\\
 To estimate the log-barrier penalty function, we use kernel density estimators, denoted by $\widehat{f}_Y(u, v;\tau)$ and $\widehat{f}_Z(w, v;\tau)$, to estimate the corresponding density functions. Hence, the estimated log-barrier function is 
 \begin{flalign*}
 \widehat{B}(\tau, \mu) = \iint - \text{sgn}(v) u \widehat{f}_Y(u, v; \tau)\,du\,dv - \mu \text{ln} \left[ \kappa - \iint \text{sgn}(v) w  \widehat{f}_{Z} (w, v; \tau) \,dw \,dv \right],
 \end{flalign*}
 and the gradient of the estimator is
 \begin{flalign*}
 \nabla \widehat{B}(\tau, \mu) = &\iint - \text{sgn}(v) u \nabla \widehat{f}_Y(u, v; \tau)\,du\,dv + \mu \frac{\iint \text{sgn}(v) w \nabla \widehat{f}_{Z} (w, v; \tau) \,dw \,dv}{\kappa - \iint \text{sgn}(v) w \widehat{f}_{Z} (w, v; \tau) \,dw \,dv}\\
  = &\iint - \text{sgn}(v) u \nabla \widehat{f}_Y(u, v; \tau)\,du\,dv + \widehat{\lambda}_{\mu} \iint \text{sgn}(v) w \nabla \widehat{f}_{Z} (w, v; \tau) \,dw \,dv,
 \end{flalign*}\\
 where $\widehat{\lambda}_{\mu}(\tau) =\sfrac{\mu}{\kappa - \iint \text{sgn}(v) w \widehat{f}_{Z} (w, v; \tau) \,dw \,dv}$.\\
 
 \textbf{Consistency of $\widehat{\tau}^k$ and $\widehat{\lambda}_{\mu}$.} \\
 We need to prove that $\widehat{\tau}^k$ is a consistent estimator of $\tau^{*k}$.\\ $\widehat{\tau}^{k} - \tau^{*k} = O_p(n^{1/2})$, and  $\widehat{\lambda}^{k} - \lambda^{*k} = O_p(n^{1/2})$. 
 \\
 Theorem proved that $\lambda_{\mu}$ is bounded.\\
 
 \textbf{Asymptotic distribution of $\widehat{\tau}^k$}\\
 Estimating equations:
  \begin{flalign*}
  \nabla \widehat{B}(\tau, \mu) = \iint - \text{sgn}(v) u \nabla \widehat{f}_Y(u, v; \tau)\,du\,dv + \hat{\lambda}_{\mu}(\tau)\iint \text{sgn}(v) w \nabla \widehat{f}_{Z} (w, v; \tau) \,dw \,dv = 0
  \end{flalign*}
 where $\hat{\lambda}_{\mu}(\tau) =  \sfrac{\mu}{ [ \kappa - \iint \text{sgn}(v) w \widehat{f}_{Z} (w, v; \tau) \,dw \,dv ]}$.\\
   \begin{flalign*}
   \nabla \widehat{B}(\tau, \mu) =& \iint - \text{sgn}(v) u \nabla \widehat{f}_Y(u, v; \tau)\,du\,dv + \hat{\lambda}_{\mu}\iint \text{sgn}(v) w \nabla \widehat{f}_{Z} (w, v; \tau) \,dw \,dv  \\
 =&-\frac{2}{n}\sum_{i=1}^{n}\boldsymbol{X}_{i,1}^{\intercal}\boldsymbol{\beta}_{Y1} k\left(-\frac{\boldsymbol{X}_{i}^{\intercal}\boldsymbol{\tau}}{h}\right) \boldsymbol{X}_{i} + \hat{\lambda}_{\mu}(\tau)\frac{2}{n}\sum_{i=1}^{n}\boldsymbol{X}_{i,1}^{\intercal}\boldsymbol{\beta}_{Z1} k\left(-\frac{\boldsymbol{X}_{i}^{\intercal}\boldsymbol{\tau}}{h}\right) \boldsymbol{X}_{i} \\
 = & N(\mu_1, \Sigma_1) + C_p N(\mu_2, \Sigma_2)
 \end{flalign*}
 
   \begin{flalign*}
   \nabla^2 \widehat{B}(\tau, \mu) =& \iint - \text{sgn}(v) u \nabla^2 \widehat{f}_Y(u, v; \tau)\,du\,dv + \\
   & \nabla\hat{\lambda}_{\mu}(\tau)\iint \text{sgn}(v) w \nabla \widehat{f}_{Z} (w, v; \tau) \,dw \,dv  + \hat{\lambda}_{\mu}(\tau)\iint \text{sgn}(v) w \nabla^2 \widehat{f}_{Z} (w, v; \tau) \,dw \,dv  \\
   = & -\frac{2}{nh}\sum_{i=1}^{n} \boldsymbol{X}_{i,1}^{\intercal}\left( \hat{\lambda}_{\mu}(\tau)\boldsymbol{\beta}_{Z1}   -\boldsymbol{\beta}_{Y1}\right) k^{\prime}\left(-\frac{\boldsymbol{X}_{i}^{\intercal}\boldsymbol{\tau}}{h}\right) \boldsymbol{X}_{i} \boldsymbol{X}^{\intercal}_{i} +\\
  & \frac{2}{n}\sum_{i=1}^{n} \nabla\hat{\lambda}_{\mu}(\tau) \boldsymbol{X}_{i,1}^{\intercal}\boldsymbol{\beta}_{Z1}   k\left(-\frac{\boldsymbol{X}_{i}^{\intercal}\boldsymbol{\tau}}{h}\right) \boldsymbol{X}_{i} 
   \end{flalign*} 
\begin{flalign*}
 \nabla\widehat{\lambda}_{\mu}(\tau) = &\frac{\mu}{\left(\kappa - \iint \text{sgn}(v) w \widehat{f}_{Z} (w, v; \tau) \,dw \,dv\right)^2} \iint \text{sgn}(v) w \nabla\widehat{f}_{Z} (w, v; \tau) \,dw \,dv \\
 = & \mu \left[\kappa -\frac{1}{n}\sum_{i=1}^{n}\boldsymbol{X}_{i,1}^{\intercal}\boldsymbol{\beta}_{Z1}\left\{ 1-2K\left(-\frac{\boldsymbol{X}_{i}^{\intercal}\boldsymbol{\tau}}{h}\right)\right\} \right]^{-1} \left[ \frac{2}{n}\sum_{i=1}^{n}\boldsymbol{X}_{i,1}^{\intercal}\boldsymbol{\beta}_{Z1} k\left(-\frac{\boldsymbol{X}_{i}^{\intercal}\boldsymbol{\tau}}{h}\right) \boldsymbol{X}_{i} \right]
\end{flalign*}
  [Notation: k and $\kappa$ looks to similar]
  
  [Need to estimate $\hat{\beta}$ too]
  

Need to prove that the difference between $\widehat{B}_n(\tau, \hat{\beta}, \mu)$ and $\widehat{B}_n(\tau, \beta^*, \mu)$ is negligible? i.e.,  $  \widehat{B}_n(\beta^*) - \widehat{B}_n(\hat{\beta}) = O_p(n^{-1/2})$\\

Taylor expansion of $\nabla \widehat{B}(\tau^{*k}, \mu)$ at $\tau = \hat{\tau}^{k}$ shows that
\begin{flalign*}
\nabla\widehat{B}(\tau^{*k}, \mu) =  \nabla\widehat{B}(\hat{\tau}^{k}, \mu) - \nabla^2\widehat{B}( \tilde{\tau}^k, \mu ) (\widehat{\tau}^k - \tau^{*k}),
\end{flalign*}
where $\tilde{\tau}^k$ is between $\tau^{*k}$ and $\widehat{\tau}^k$. As $\hat{\tau}^{k}$ is the minimizer of $B(\tau, \mu)$, it satisfies the first order condition that $\nabla B(\hat{\tau}^k, \mu) = 0$ . Therefore, we have 
\begin{flalign*}
\sqrt{n}\nabla\widehat{B}(\tau^{*k}, \mu) =   - \sqrt{n} \nabla^2\widehat{B}( \tilde{\tau}^k, \mu ) (\widehat{\tau}^k - \tau^{*k}).
\end{flalign*}

\textbf{Derivation of the integrations}\\
The integration we need
\begin{gather*}
\begin{flalign*}
 & \iint \text{sgn}\left( v \right) u \,f\left( u,v ;\tau, \beta_{\cdot1} \right) \,du\,dv \\
= & 2\iint u\,\mathbb{I}\left(v\ge0\right)f\left(v,u;\tau,\beta_{\cdot1}\right)\,dv\,du-\int u\,f\left(u;\beta_{\cdot1}\right)\,du
\end{flalign*}
\end{gather*}
The estimator is
\begin{gather*}
\begin{flalign*}
& \iint \text{sgn}\left( v \right) u \,\widehat{f}_n\left( u,v ;\tau, \beta_{\cdot1} \right) \,du\,dv \\
= & 2\iint u\,\mathbb{I}\left(v\ge0\right)\widehat{f}_n\left(v,u;\tau,\beta_{\cdot1}\right)\,dv\,du-\int u\,\widehat{f}_n\left(u;\tau, \beta_{\cdot1}\right)\,du \\
=  & \frac{2}{nh^2}\iint u\,\mathbb{I}\left(v\ge0\right)\sum_{i=1}^{n}k\left(\frac{v-V_{i}}{h}\bigg)k\bigg(\frac{u-U_i}{h}\right)\,du\,dv-\\
&\frac{1}{nh}\int u\sum_{i=1}^{n}k\left(\frac{u-U_{i}}{h}\right)\,du\\
= & \frac{2}{n}\sum_{i=1}^{n}\boldsymbol{X}_{i,1}^{\intercal}\boldsymbol{\beta}_{\cdot1}\left\{ 1-K\left(-\frac{\boldsymbol{X}_{i}^{\intercal}\boldsymbol{\tau}}{h}\right)\right\} -\frac{1}{n}\sum_{i=1}^{n}\boldsymbol{X}_{i,1}^{\intercal}\boldsymbol{\beta}_{\cdot1}\\
=& \frac{1}{n}\sum_{i=1}^{n}\boldsymbol{X}_{i,1}^{\intercal}\boldsymbol{\beta}_{\cdot1}\left\{ 1-2K\left(-\frac{\boldsymbol{X}_{i}^{\intercal}\boldsymbol{\tau}}{h}\right)\right\}
\end{flalign*}
\end{gather*}
where $\widehat{f}_n(u_{1},u_{2};\tau,\widehat{\beta}_{\cdot1})$
are the kernel density estimator for $\left(X^{\intercal}\tau, X^{\intercal}\beta_{\cdot1}\right)$
with the forms of 
\begin{gather*}
\widehat{f}_n(u,v;\boldsymbol{\tau},\widehat{\boldsymbol{\beta}}_{\cdot1})=\frac{1}{nh^2}\sum_{i=1}^{n}k\left(\frac{u-U_{i}}{h}\right)k\left(\frac{v-V_{i}}{h}\right).
\end{gather*}
Moreover, $K(s)$ is the corresponding CDF of the kernel function $k(s)$, which is chosen to be a symmetric probability density. More precisely, $k(s)$ satisfies the following assumptions:
\begin{enumerate}
	\item $\int_{-\infty}^{\infty}k(s)\,ds=1.$
	\item $k(s)>0$ for all $s$.
	\item $k(-s)=k(s)$ for all $s$. 
	\item The first order derivative of the kernel, $k^{\prime}(s)$,
	exists and is bounded. 
\end{enumerate}
The last equality above holds by following the derivation.\\

We first derive $\frac{2}{h^2}\iint u_{2}\,\mathbb{I}\left(u_{1}\ge0\right)k\left(\frac{u_{1}-U_{i,1}}{h}\bigg)k\bigg(\frac{u_{2}-U_{i,2}}{h}\right)\,du_{1}\,du_{2}$.
Let $s=\frac{u_{1}-U_{i,1}}{h}$ and $t=\frac{u_{2}-U_{i,2}}{h}$.
Then, $u_{1}=U_{i,1}+sh$ and $u_{2}=U_{i,2}+th$. Also,
$du_{1}=h\,ds$ and $\,du_{2}=h\,dt$. 
\begin{flalign*}
&\frac{2}{h^2}\iint u_{2}\,\mathbb{I}\left(u_{1}\ge0\right)k\left(\frac{u_{1}-U_{i,1}}{h}\bigg)k\bigg(\frac{u_{2}-U_{i,2}}{h}\right)\,du_{1}\,du_{2}\\
= & 2\iint\left(U_{i,2}+th\right)\,\mathbb{I}\left(U_{i,1}+sh\ge0\right)k\left(s\right)k\left(t\right)\,ds\,dt\\
= & 2\int U_{i,2}\,\mathbb{I}\left(s\ge-\frac{U_{i,1}}{h}\right)k\left(s\right)\,ds\\
= & 2U_{i,2}\left\{ 1-K\left(-\frac{U_{i,1}}{h}\right)\right\} \\
= & 2\boldsymbol{X}_{i,1}^{\intercal}\boldsymbol{\beta}_{\cdot1}\left\{ 1-K\left(-\frac{\boldsymbol{X}_{i}^{\intercal}\boldsymbol{\tau}}{h}\right)\right\} ,
\end{flalign*}
where $K\left(s\right)=\int k\left(s\right)\,ds+c$. The
second equality holds, as $\int k(t)\,dt=1$ and $\int t\,k(t)\,dt=0$.
The third equality holds as $\int\mathbb{I}\left(s\ge-\frac{U_{i,1}}{h}\right)k\left(s\right)\,ds=1-\int_{-\infty}^{-\sfrac{U_{i,1}}{h}}k\left(s\right)\,ds=1-K\left(-\frac{U_{i,1}}{h}\right)$,
where $U_{i,1}=\boldsymbol{X}_{i}^{\intercal}\boldsymbol{\tau}$.\\

Then, we derive $\frac{1}{h}\int u_{2}k(\frac{u_{2}-U_{i,2}}{h})\,du_{2}$
by changing variable similarly. Let $t=\frac{u_{2}-U_{i,2}}{h}$
,and we get $u_{2}=U_{i,2}+th$, and $\,du_{2}=h\,dt$.
\begin{align*}
&\frac{1}{h}\int u_{2}k\left(\frac{u_{2}-U_{i,2}}{h}\right)\,du_{2}\\
= & \int\left(U_{i,2}+th\right)k\left(t\right)\,dt\\
= & U_{i,2}
= \boldsymbol{X}_{i,1}^{\intercal}\boldsymbol{\beta}_{\cdot1}.
\end{align*}
Again, the second equality holds as $\int k(t)\,dt=1$,
and $\int t\,k(t)\,dt=0$.

The integration over the first-order derivative 
\begin{flalign*}
& \iint \text{sgn}\left( v \right) u \, \nabla\widehat{f}_n\left( u,v ;\tau, \beta_{\cdot1} \right) \,du\,dv \\
= & \frac{\partial}{\partial \tau} \iint \text{sgn}\left( v \right) u \, \widehat{f}_n\left( u,v ;\tau, \beta_{\cdot1} \right) \,du\,dv\\
=&\frac{2}{n}\sum_{i=1}^{n}\boldsymbol{X}_{i,1}^{\intercal}\boldsymbol{\beta}_{\cdot1} k\left(-\frac{\boldsymbol{X}_{i}^{\intercal}\boldsymbol{\tau}}{h}\right) \boldsymbol{X}_{i}
\end{flalign*}
  \end{document}