% modeling conditional distributions
\documentclass[../main.tex]{subfiles}
\begin{document}
\textbf{\large Reformalize the problem using penalty barrier}\\
Barrier penalty function is used to re-formalize  Problem (1), with additional restrictions on both the euclidean norms of $\bs{\tau}_1$ and $\bs{\tau}_2$ to be 1 for identification, for a decreasing sequence $\{\mu\}$, where $\mu \to 0$,
\begin{equation}
\underset{\bs{\tau}}{\max }\,\, \mb{E} Y^*(\bs{\tau}) + \mu \txt{log} \lt\{ \kappa - \mb{E} Z^*(\bs{\tau}) \rt\} - \frac{1}{2\mu} \lt\{ (\bs{\tau}_1^\itl \bs{\tau}_1 -1 )^2 + (\bs{\tau}_2^\itl \bs{\tau}_2 -1 )^2 \rt\}
\end{equation}

Let $\bs{\tau}^{*\itl}_{\kappa}(\mu) = \{\bs{\tau}^{*\itl}_{\kappa,1}(\mu), \bs{\tau}^{*\itl}_{\kappa,2}(\mu)\}$ denote a solution to Problem (2), and $\bs{\tau}^{*\itl}_{\kappa,\mu} = \{\bs{\tau}^{*\itl}_{\kappa,\mu, 1}, \bs{\tau}^{*\itl}_{\kappa, \mu, 2}\}$ for short. The corresponding regime is denoted by $\bs{d}^*_{\kappa}(\mu) = \lt\{d^*_{1,\kappa}(\mu), d^*_{2,\kappa}(\mu)\rt\}$, and $\bs{d}^*_{\kappa,\mu} = \lt\{d^*_{\kappa,\mu,1}, d^*_{\kappa,\mu,2}\rt\}$ for short. Then, $d^*_{\kappa,\mu,1}(\bs{h}_1) = \tsgn(\bs{h}^\itl_1 \bs{\tau}^*_{\kappa,\mu,1})$, and $d^*_{\kappa,\mu,2}(\bs{h}_2) = \tsgn(\bs{h}^\itl_2 \bs{\tau}^*_{\kappa,\mu,2})$. For short notation, we let $S^*(\bs{\tau}, \mu) = \mb{E} Y^*(\bs{\tau}) + \mu \txt{log} \lt\{ \kappa - \mb{E} Z^*(\bs{\tau}) \rt\} - \frac{1}{2\mu} \lt\{ (\bs{\tau}_1^\itl \bs{\tau}_1 -1 )^2 + (\bs{\tau}_2^\itl \bs{\tau}_2 -1 )^2 \rt\}$.  \\

\textbf{\large Convergence of the Penalty-Barrier Trajectory  $\bs{\tau}^*_{\kappa}(\mu)$  to $\bs{\tau}^0_{\kappa}$}\\

We again examine the conditions under which the penalty-barrier trajectory $\bs{\tau}^*_{\kappa}(\mu)$ converges to the original constrained maximizer $\bs{\tau}^0_{\kappa}$, of which details are provided Appendix 1 Proof Draft 1. We specify  the conditions needed for our problem as follows. For notation simplicity in this section, we let $f(\bs{\tau}) = \mb{E} Y^*(\bs{\tau})$, $c_1 (\bs{\tau}) = \kappa - \mb{E}Z^*(\bs{\tau})$, $c_2(\bs{\tau}) = \bs{\tau}_1^\itl \bs{\tau}_1 - 1$, and $c_3(\bs{\tau}) = \bs{\tau}_2^\itl \bs{\tau}_2 - 1$. Let $g(\bs{\tau})$ denote the gradient of $f(\bs{\tau})$, i.e., $g(\bs{\tau}) = \nabla f(\bs{\tau}) = \nabla \mb{E}Y^*(\bs{\tau})$. Also, let $\bs{c}(\bs{\tau})$ be the vector of constraint functions $\{c_i(\bs{\tau})\}$, $i = 1, 2, 3$. The Jacobian matrix $\bs{c}^{\prime}(\bs{\tau})$ of first derivative of $\bs{c}(\bs{\tau})$ has row $\{\nabla c_{i}(\bs{\tau})\}^{\itl}$, and we use $J(\bs{\tau})$ to denote this Jacobian for concise.\\
\begin{lemma}[Conditions for the trajectory $\{ \bs{\tau}_{\kappa}^*(\mu) \}$ converging to $\bs{\tau}^0_{\kappa}$\cite{Nocedal1999,fiacco,Forsgren2002}]
Assume
\begin{enumerate}
		\item the functions $f$, $c_1$, $c_2$, and $c_3$ are twice differentiable with respect to $\bs{\tau}$;
		\item the gradients $\nabla c_1$, $\nabla c_2$, and $\nabla c_3$ are linearly independent, where the gradients are taken with respect to $\bs{\tau}$;
		\item strict complementarity holds for  $\lambda_1^0 c_1(\bs{\tau}_{\kappa}^0) = 0$, where $\lambda_1^0$ is the Lagrangian multiplier of the inequality constraint $c_1$;
		\item the sufficient conditions under which $\bs{\tau}_{\kappa}^0$ be an isolated local constrained minimum of Problem (2) are satisfied by $(\bs{\tau}^0_{\kappa}, \bs{\lambda}^{0})$, where $\bs{\lambda}^0 = (\lambda_1^0, \lambda_2^0, \lambda_3^0)^\itl $ of which $\lambda_2^0$ is the  Lagrangian multipliers for the  constraint $c_2$, and $\lambda_3$ for $c_3$. The sufficient conditions for optimality are
		\begin{enumerate}
			\item $\bs{\tau}_{\kappa}^0$ is feasible and the LICQ (Linear Independence Constraint Qualification) holds at $\bs{\tau}_{\kappa}^0$, i.e., the Jacobian matrix of active constraints at $\bs{\tau}_{\kappa}^0$, $J_{\mathcal{A}}(\bs{\tau}_{\kappa}^0)$, has full row rank;
			\item $\bs{\tau}_{\kappa}^0$ is a KKT point and strict complementarity holds, i.e, the (necessarily unique) multiplier $\bs{\lambda}^0$ has the property that $\lambda_i^0 > 0$, for all $i  \in \mathcal{A}_{\mathcal{I}}(\bs{\tau}_{\kappa}^0)$, the set of indices of active inequality constraints at $\bs{\tau}_{\kappa}^0$;
			\item for all nonzero vectors $\bs{p}$ satisfying $J_{\mathcal{A}}(\bs{\tau}_{\kappa}^0)\bs{p} = 0$, there exists $\omega > 0$ such that $\bs{p}^{\intercal}H(\bs{\tau}_{\kappa}^0, \bs{\lambda}^0) \bs{p} \ge \omega \|\bs{p}\|^2$., where $H(\bs{\tau}_{\kappa}^0, \bs{\lambda}^0) $ is the hessian of the Lagrangian at $\bs{\tau}_{\kappa}^0$ and $\bs{\lambda}^0$, where $\bs{\lambda}^0$ is the vector of the Lagrangian multipliers, $\bs{\lambda}^0 = (\lambda_1^0, \lambda_2^0, \lambda_3^0)^{\intercal}$.
		\end{enumerate} 
		then there is a positive neighborhood about $\mu = 0$ for which a unique-isolated differentiable function $\bs{\tau}^*_{\kappa}(\mu)$ exists. It describes a unique isolated trajectory of local maxima of $S^*(\bs{\tau}, \mu)$, where $\bs{\tau}^*_{\kappa}(\mu) \to \bs{\tau}^0_{\kappa}$ as $\mu \to 0$.
	\end{enumerate}
\end{lemma}

To find $\bs{\tau}^*_{\kappa}(\mu)$, we need to examine its stationarity. That is $\nabla S^*(\bs{\tau}, \mu) = 0$ is satisfied at $\bs{\tau}^*_{\kappa}(\mu)$. Its equivalent system of non-linear equations is
\begin{align*}
F^{\mu}(\bs{\tau}, \bs{\lambda}) = 
\begin{pmatrix} g(\bs{\tau}) - J(\bs{\tau}) \bs{\lambda} \\ c_1( \bs{\tau}) \lambda_1 - \mu \\ c_2(\bs{\tau}) + \mu \lambda_2\\ c_3(\bs{\tau}) + \mu \lambda_3 \end{pmatrix} = 0
\end{align*}
We also define $\chi_1 \triangleq \sfrac{\mu}{c_1(\bs{\tau})}$, $\chi_2 \triangleq - \sfrac{c_2(\bs{\tau})}{\mu}$ and $\chi_3 \triangleq - \sfrac{c_3(\bs{\tau})}{\mu}$ which can be considered as approximates of the Lagrangian multipliers under  $\mu$-perturbed KKT conditions. \\

\begin{comment}
Is $d F - d \hat{F} = d(F - \hat{F})$ ??\\
Note: 
convergence $\hat{\tau}_{\kappa}(\mu) \overset{p}{\to} \tau^*_{\kappa}(\mu) \to  \tau^0_{\kappa}(\mu)$ 
\begin{flalign*}
&  \hat{\mb{E}} Y(\bs{\tau}) - \mb{E} Y^*(\bs{\tau})  \\
= & \int y \,d \wh{F}_{Y_d} (y) - \int y \,d F_{Y^*_d} (y) \\
= &  \int y  \lt\{  \,d\wh{F}_{Y_d} (y) - \,d F_{Y^*_d} (y) \rt\}  \\
% = & \int y  \lt\{ \wh{f}_{Y_d} (y) - f_{Y^*_d} (y) \rt\} \,dy
\end{flalign*}
If for any arbitrary $\bs{d}$, $d F_{Y^*_{\bs{d}}} (y) $ is a density and $d\wh{F}_{Y_{\bs{d}}}(y)$ is a uniformly consistent estimator; that is $\underset{y}{\sup} \mid d\wh{F}_{Y_d} (y) - d F_{Y^*_d} (y) \mid =o_p(1)$, then 

\begin{flalign*}
&  \hat{\mb{E}} Y(\bs{\tau}) - \mb{E} Y^*(\bs{\tau})  \\
= & \int y \,d \wh{F}_{Y_d} (y) - \int y \,d F_{Y^*_d} (y) \\
= &  \int y  \lt\{  \,d\wh{F}_{Y_d} (y) - \,d F_{Y^*_d} (y) \rt\}  \\
\le & \int \mid y \mid \, dy  \cdot o_p(1) \\
% = & \int y  \lt\{ \wh{f}_{Y_d} (y) - f_{Y^*_d} (y) \rt\} \,dy\\
= &  \iint  F_{\varepsilon_Y}\lt[ y - m(\bs{h}_2) - \tsgn\lt\{r_2(\bs{h}_2; \bs{\tau}_2)\rt\}c_Y(\bs{h}_2) \rt] \,d G_{Y}\lt\{ m_Y, c_Y, r_2 \rvert \bs{h}_1 , d_1(\bs{h}_1)\rt\} \,d F_{\bs{H}_1}(\bs{h}_1) 
\end{flalign*}

\textbf{Reference}\\

\textbf{Convergence in Distribution}\\
Suppose that $(X_1, X_2, \dots)$ and $X$ are real-valued random variables with distribution functions $(F_1, F_2,  \dots)$ and $F$, respectively. We say that the distribution of $X_n$ converges to the distribution of $X$ as $n \to \infty$ if
$$F_n(x) \to F(x), \text{as } n \to \infty$$
for all $x$ at which $F$ is continuous. \\
% Link https://www.probabilitycourse.com/chapter7/7_2_4_convergence_in_distribution.php

\textbf{Lebesgue's Dominated Convergence Theorem}\\
If for some random variable $Z$, $| X_n | \le | Z |$ for all $n$ and $\mb{E} | Z | \le \infty$, then $X_n \overset{d}{\to} X$ implies that $\mb{E} X_n \to \mb{E} X$.

\end{comment}


\end{document}